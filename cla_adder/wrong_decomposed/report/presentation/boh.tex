\documentclass{beamer}
\usetheme{Warsaw}
%Warsaw,Singapore, 
%\usecolortheme{whale}

\usepackage{graphicx}
\newcommand{\unit}[1]{\ensuremath{~\mathrm{#1}}} %usate questo commando nuovo per scrivere le unitá di misure dentro al math environment
\graphicspath{{./figures//}}

\title{Chaos in Electronics and Telecommunication}
\subtitle{Analog and Telecommunication Electronics\\
mini-Project\\}
\institute{Politecnico di Torino}
\author{Daniel Rodas Bautista\\
219976\\}
%\setbeamertemplate{footline}{\insertframenumber/\inserttotalframenumber}
\useoutertheme{infolines}
\begin{document}

\frame{\titlepage}

\begin{frame}
        \frametitle{Introduction}
        %first slide
        \begin{itemize}
                \item Chaos theory has arisen from the study of dynamic non-linear systems. 
                \item Nonlinearity gives rise to many interesting features such as inter-modulation. 
                \item Under certain condition some nonlinear systems  present chaotic behavior.
                \item In recent years there has been a lot of interest in Chaos theory. Even more recently some application in electronics have been explored.
        \end{itemize}

\end{frame}
